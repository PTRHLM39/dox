\documentclass[../main.tex]{subfiles}

\begin{document}

\subsubsection{Exekveringsfel}
\label{sec:exec}
\begin{lstlisting}[language=c]
#include <stdio.h>

int main() {

    char file_name[100];

    printf("Enter file-name: ");
    scanf("%s", file_name);

    FILE *input_file = fopen(file_name, "r");

    char s[100];
    while (fgets(s, 100, input_file) != NULL)
        printf("%s", s);
\end{lstlisting}

\begin{itemize}

    \item I programmet ovan tilldelas en sträng-variabeln(tecken-fältet) \texttt{file\_name} ett filnamn av användaren.
    
    \item Sedan tilldelas \texttt{input\_file} av typen \texttt{FILE} till att öppna filen med namnet \texttt{file\_name} som tilldelades av användaren. "r" beskriver att \texttt{fopen()} skall läsa(read) den valda filen.
    
    \begin{lstlisting}[language=c]
    FILE *input_file = fopen(file_name, "r");
    \end{lstlisting}
    
    \item Sedan tilldelas tecken-fältet \texttt{s}. och en \texttt{while}-loop deklareras till att läsa de 100 första tecknen i filen och skriva ut i terminal-fönstret.
    
    \begin{lstlisting}[language=c]
    char s[100];
    while (fgets(s, 100, input_file) != NULL)
        printf("%s", s);
    \end{lstlisting}
    
\end{itemize}

\begin{tcolorbox}[colback=green!5!white,colframe=green!75!black]
  Tack för att du läste. Gå tillbaks till början \hyperref[sec:top]{\textbf{här}}
\end{tcolorbox}

\end{document}