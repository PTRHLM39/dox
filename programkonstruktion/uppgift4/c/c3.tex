\documentclass[../main.tex]{subfiles}

\begin{document}
\subsection{Exekveringsfel}
\label{sec:c3}
Detta fel uppstår under exekveringen av program. Programmet följer dess språkregler och kompileras felfritt, men innehåller fel, som vid mänsklig inmatning gör att programmet genererar en felsignal, och om detta scenario inte hanteras i programmet, avbryts exekveringen med en felutskrift.\\
\\
Följande är ett program som läser in en vald textfil, och skriver ut dess innehåll i terminal-fönstret.
\begin{lstlisting}[language=c]
#include <stdio.h>

int main() {

    char file_name[100];

    printf("Enter file-name: ");
    scanf("%s", file_name);

    FILE *input_file = fopen(file_name, "r");

    char s[100];
    while (fgets(s, 100, input_file) != NULL)
        printf("%s", s);
}
\end{lstlisting}
Ett exekveringsfel som kan uppstå vid körning av detta program, är att filen som skall öppnas inte existerar i arbetskatalogen, där programtexten ligger.\\
\\
Då uppstår felutskrifter som,
\begin{lstlisting}
Enter file-name: .bash_history
Segmentation fault (core dumped)
\end{lstlisting}
och programmet avbryts.
\\
För att undvika att programmet avbryts och lämnar användaren i oklarheter, om vad som skett.
Kan man fånga felet och skriva ut en förklaring till varför felet uppstod:

\newpage

\begin{lstlisting}[language=c]
#include <stdio.h>
#include <stdlib.h>

int main() {
    char file_name[100];

    printf("Enter file-name: ");
    scanf("%s", file_name);

    FILE *input_file = fopen(file_name, "r");

    if (input_file == NULL) {
        printf("Can't find any file named '%s'", file_name);
        exit(1);
    }

    char s[100];
    while (fgets(s, 100, input_file) != NULL)
        printf("%s", s);
}
\end{lstlisting}
if-satsen som lades till, kollar ifall det tilldelade filnamnet är kopplad till någon befintlig fil.\\
Om så inte är fallet får man utskriften;
\begin{lstlisting}
Enter file-name: .bash_history
Can't find any file named .bash_history
\end{lstlisting}

\begin{tcolorbox}[colback=green!5!white,colframe=green!75!black]
Referens till programtext finner du \hyperref[sec:exec]{\textbf{här}}
\end{tcolorbox}

\end{document}