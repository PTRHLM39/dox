\documentclass[../main.tex]{subfiles}

\begin{document}

\section{C-uppgift}
\label{sec:top}
Det kan finnas tre olika typer av fel i ett program:

\begin{itemize}
    \item Kompileringsfel
    \item Logiska fel
    \item Exekveringsfel
\end{itemize}

Förklara dessa typer med hjälp av tre olika exempel (källkod).
OBS! Detta är en C-uppgift. Du behöver förklara din källkod i detalj (Utförligt).

\subsection{Kompileringsfel}

Varje programmeringsspråk har sina egna uppsättningar av regler för att kunna formulera sina texter. 
Programmerare måste följa respektive skrivregler för att datorn skall kunna förstå programmen vid kompilering.
Att lyckas följa dessa regler fullt ut är inte alltid lätt, och nästintill överraskande om man lyckas på första försöket. Misslyckas man får man snart reda på vad som inte stämmer genom kompileringsfelet som dyker upp i terminalfönstret. Dessa fel kallas kompileringsfel och uppstår vid kompileringen av program.\\
Därmed är det lätt att finna och rätta dessa fel.\\

Nedan följer ett exempel på hur en mängd kompileringsfel kan se ut vid kompilering av ett program.

\begin{lstlisting}[language=c]
#include <stdio.h> 

int main () 
{
    int value; 
    
    printf("Assign value: "; 
    
    scanf("%d", value); 
    
    Double root = sqrt(value) 
    
    printf("Squareroot of %d = %f\n", value root);
 
}
\end{lstlisting}

\newpage

Vilket ger kompileringsfelen:
\begin{lstlisting}
square.c: In function 'main':
square.c:6:28: error: expected ')' before ';' token
     printf("Assign value: ";
                            ^
s.c:12:1: error: expected ';' before '}' token
 }
 ^
\end{lstlisting}

\begin{itemize}

        
    \item Som visar att filen vi precis kompilerade (filen square.c) har en funktion/klass som är namngiven "main" och att inuti denna funktion befinner sig ett eller fler kompileringsfel.
    
        \begin{lstlisting}
            square.c: In function 'main':
        \end{lstlisting}

    \item Sedan en riktning, genom vilken rad och vid vilket tecken problemet har uppstått i. Följt av ett felmeddelande på hur problemet ser ut.
    
        \begin{lstlisting}
            square.c:6:28: error: expected ')' before ';' token
                 printf("Assign value: ";
                            ^
        \end{lstlisting}


\end{itemize}


Där ifrån kan man sedan se i källkoden att på rad 6 vid tecken 28 skall det vara en parentes ')' innan semikolon (;).
\begin{lstlisting}[language=c]
    printf("Assign value: "; // *Ledsen-emoji*
    printf("Assign value: "); // *Glad-emoji* 
\end{lstlisting}

\newpage

Fortsätt sedan med nästa kompileringsfel i listan tills programmet lyckas kompileras och får utseendet:

\begin{lstlisting}[language=c]
#include <stdio.h>
#include <math.h>

int main ()
{
    int value;
    
    printf("Assign value: ");
    scanf("%d", value); 
    
    double root = sqrt(value);
    
    printf("Squareroot of %d = %f\n", value root);
}
\end{lstlisting}
Med utmatningen:
\begin{lstlisting}
Assign value: 2                                                                                                                           
Squareroot of 2 = 1.414214
\end{lstlisting}

\begin{tcolorbox}[colback=green!5!white,colframe=green!75!black]
 Referens till programtext finner du \hyperref[sec:compile]{\textbf{här}}
\end{tcolorbox}


\newpage
\end{document}