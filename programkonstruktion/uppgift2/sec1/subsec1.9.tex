\documentclass[../main.tex]{subfiles}

\begin{document}

\subsection{Ge några exempel på grundläggande numeriska operatorer?}
\textbf{Exempel}
\begin{lstlisting}[language=java]
int a = 10, b = 20;
\end{lstlisting}
\begin{tabular}{ |p{3cm}|p{3cm}|p{3cm}| }
 \hline
 \multicolumn{3}{|c|}{\texttt{Numeriska operatorer i Java}} \\
 \hline
 \textbf{Operator}& \textbf{Beskrivning}& \textbf{Exempel}\\
 \hline
 + (addition) & Adderar värden & a + b = 30 \\
 \hline
 - (subtraktion) & Subtraherar värden & a - b = -10 \\
 \hline
 * (multiplikation) & Multiplicerar värden & a * b = 200 \\
 \hline
 / (division) & Dividerar värden & b / a = 2 \\
 \hline
 \% (modulus) & Delar vänsteroperanden med högeroperanden och returnerar resten. & b \% a = 0 \\
 \hline
 ++ (ökning) & Ökar värdet på operanden med 1 & a++ ger 11 \\
 \hline
 - -  (minskning) & Minskar värdet på operanden med 1 & a- - ger 9 \\
 \hline
 
\end{tabular}
\end{document}