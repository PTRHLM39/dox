\documentclass[../main.tex]{subfiles}

\begin{document}

\subsection{Vad händer vid explicit typomvandling?}
När man tilldelar värden från en data-typ till en annan, kan dessa data-typer inte vara kompatibla med varandra. Är typerna däremot kompatibla med varandra utför Java en automatiskt typomvandling. Om inte detta går behöver programmeraren manuellt deklarera explicit typomvandling mellan data-typerna.\\

Automatisk typomvandling sker när data-typerna;
\begin{itemize}
    \item Data-typerna är kompatibla.
    \item När man tilldelar en mindre data-typ till en större.
\end{itemize}
\begin{lstlisting}[language=java]
// byte -> short -> int -> long -> float -> double
// Automatisk typomvandling

// Exempel
int i = 10;
long l = i; // Automatisk.

float f = l // automatisk

/** Utmatning:
 * int:   10
 * long:  10
 * float: 10.0
 */
\end{lstlisting}
Vill man tilldela en data-typ av större sort måste man använda explicit typomvandling.
\begin{lstlisting}[language=java]
// double -> float -> long -> int -> short -> byte

// Exempel
double d = 50.2;

long l = (long)d; // (datatyp)variabel explicit typomvandling
int i = (int)l;

/** Utmatning:
 * double: 50.2
 * long:   50
 * int:    50
 */
\end{lstlisting}
\end{document}