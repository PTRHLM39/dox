\documentclass[../main.tex]{subfiles}

\begin{document}

\subsection{Ge några exempel på tilldelningssatser.}

\begin{lstlisting}[language=java]
double pi = 3.14159;

String text = "35.1";
double radius = Double.parseDouble(text); // 35.1

int a, b = 2, c = 3;
a = b + c; // a = 5
\end{lstlisting}

\subsection{Vad menas med att "initiera variabler"?}
Att initiera en variabel är att tilldela ett värde samtidigt man deklarerar variabeln i programmet. Det är viktigt att tilldela en variabel ett värde innan programmet försöker avläsa vad variabeln innehåller. Kompilatorn upptäcker ifall detta skulle ske och ger ut felmeddelanden som;
\begin{lstlisting}
Info.java:10 error: variable name might not have been initialized
               contactInfo = name + "\n" + adress;
                               ^
1 error
\end{lstlisting}
Att initiera en variabel ser ut som följande;

\begin{lstlisting}[language=java]
String text = "Java-programmering";
int heltal = 12;
double flyttal = 2.15;
\end{lstlisting}
\end{document}