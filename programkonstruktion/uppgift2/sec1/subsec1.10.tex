\documentclass[../main.tex]{subfiles}

\begin{document}

\subsection{Vad menas med prioriteringsordning? Ge några exempel.}
Som den ordinära matematikens prioriteringsregler, gäller även vid programmeringens aritmetik. Operatorerna \textbf{*} (multiplikation) och \textbf{/} (division) har högre prioritet än \textbf{+} (addition) och \textbf{-} (subtraktion).\\
\\
Om ett numeriskt uttryck innehåller fler operatorer utförs den operatorn med högst prioritet först. Har operatorerna samma prioritetgrad utförs uträkningen helt enkelt från vänster till höger. \\
Man kan skifta ordningen med parenteser precis som i vanlig matematik.
\\
Exempel;
\begin{lstlisting}[language=java]
int x = 5, y = 3;
x + 2 * y;   // Blir 11 i och med multiplikationens prioritet.

(x + 2) * y; // Blir 21 i och med parentesernas inverkan.

x * 2 * y;   // Blir 30.

x * (2 * y); // Blir 30.
\end{lstlisting}
Följande har inget med java att göra men finner det intressant hur det kan skilja mellan olika programspråk.\\
Programmeringsspråk som Lisp-språken använder sig av "omvänd polsk notation" för att uttrycka prioriteringsreglerna.
I "omvänd polsk notation" skrivs operatorerna före operanderna.\\
Exempel;
\begin{lstlisting}[language=lisp]
;;; Aritmetik i Common-lisp

(let ((x 5)  ; Variabel initiering
      (y 3)) ; *******************
      (+ x (* 2 y))    ; 11
      (* (+ x 2) y)    ; 21
      (* x 2 y)        ; 30
      (* (* 2 y) x))   ; 30
\end{lstlisting}
\end{document}