\documentclass[../main.tex]{subfiles}

\begin{document}

\subsection{Ge tre olika exempel på logiska uttryck.}

\textit{När man skall välja mellan olika vägar i ett program använder man för det mesta en if-sats.}
\textit{Denna sats kan se ut på olika sätt. Den enklaste versionen har formen:}
\begin{lstlisting}[language=java]
if (logiskt uttryck)
    sats;
\end{lstlisting}

\textbf{Exempel på logiska uttryck:}

\begin{lstlisting}[language=java]
// Exempel 1
if (liquidLevel >= maxLevel)
    openValve();
    
// Exempel 2
if (liquidLevel < minLevel) {
    closeValve();
    fillReservoir();
}
// Exempel 3
if (waterTemp > thermostatValue || waterTemp < thermostatValue)
    adjustWaterTemp();
    
\end{lstlisting}

\end{document}