\documentclass[../main.tex]{subfiles}

\begin{document}

\subsection{Konvertera följande while-sats till en for-sats, förklara koden i detalj
och beskriv vad blir utskriften blir efter körning (efter lämplig komplettering av koden).}

\begin{lstlisting}[language=java]
int k = 0; // Heltalsvariabeln k, initieringsdel.
String s = "k:"; // Textvariabeln s 
while(k < 6) {  // Logiskt uttryck, villkorsdel.
    s = s + " " + k; // sats1
    k = k + 2; // sats2, forandringsdel.
    
    System.out.println(s); // Utmatning i kommandotolken
}
\end{lstlisting}
\begin{itemize}
    \item Heltalsvariabeln \textbf{k}
    sätts som räknare till \textbf{while}-satsens logiska uttryck. \textbf{while}-satsens uppgift är att repetera en eller flera programsatser, för var gång \textbf{while}-satsens logiska uttryck är sant, utförs satserna innanför \textbf{while}-satsens klamrar. I det här fallet utförs \textbf{while}-satsen 3 gånger tills det logiska uttrycket är falskt och inga fler varv i \textbf{while} utförs.
    
    \item Sats1 innanför klamrarna tilldelar strängvariabeln \textbf{s}, strängen, "\texttt{k: <whitespace>}" samt värdet i "\texttt{k}" för varje varv \textbf{while}-satsen utförs.
    
    \item Sats2 utökar värdet genom att addera värdet 2 till heltalsvariabeln \textbf{k} för varje varv i \textbf{while}-satsen.
    
    \item Genom att placera en \textbf{println}-metod innuti \textbf{while} kan vi se hur det hela successivt byggs upp genom repetitionens gång, via kommandotolken. Efter exekvering får vi utmatningen;
    
    \begin{lstlisting}
    k:  0
    k:  0 2
    k:  0 2 4
    \end{lstlisting}
    Ändrar vi det logiska uttrycket till \textbf{while}(k $<$ 12), får vi utmatningen;
    \begin{lstlisting}
    k:  0
    k:  0 2
    k:  0 2 4
    k:  0 2 4 6
    k:  0 2 4 6 8
    k:  0 2 4 6 8 10
    \end{lstlisting}
    Utförs istället \textbf{while} 6 gånger.
\end{itemize}
\newpage

Nedan följer \textbf{while}-satsen tidigare, omskriven till en \textbf{for}-loop.
\begin{lstlisting}[language=java]
String s = "k:";

for (int k = 0; k < 12; k = k + 2) {
    s = s + " " + k;
    System.out.println(s);
}
// Utmatning:
// k: 0
// k: 0 2
// k: 0 2 4
// k: 0 2 4 6
// k: 0 2 4 6 8
// k: 0 2 4 6 8 10
\end{lstlisting}
Skillnaden mellan en \textbf{while} och en \textbf{for}-loop är att inuti \textbf{for}-loopens parenteser sker en initieringsdel
en vilkorsdel(logisk uttryck) samt en förändringsdel.

\begin{itemize}
    \item \textbf{while}-loopens initieringsdel sker innan repetitions-satsen.
    \item \textbf{while}-loopens villkorsdel sker efter \textbf{while}-deklarationen, inuti dess parenteser. Likt \textbf{for}.
    \item \textbf{while}-loopens förändringsdel sker inuti dess klamer-kropp.
\end{itemize}
\end{document}