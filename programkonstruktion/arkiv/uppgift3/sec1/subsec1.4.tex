\documentclass[../main.tex]{subfiles}

\begin{document}

\subsection{Förklara skillnaden mellan jämförelseoperatorer och logiska operatorer med hjälp av några exempel.}

Används jämförelseoperatorer($<$, $>$, != etc) så bildas uttryck av data-typen boolean.

\begin{lstlisting}[language=java]
int a = 5, b = 10;
/*Lika med*/
a == b; // falskt
/*Icke lika med */
a != b // sant
/* less than */
a < b // sant
/* Greater than */ 
a > b // falskt

\end{lstlisting}
Det finns även tre andra operatorer som används till att bilda mer advancerade logiska uttryck, dessa heter logiska operatorer.
\begin{lstlisting}[language=java]
// && och-operator, || eller-operator, ! icke-operator.
int a = 5, b = 10;


a == 5 || b == 20; // sant 
a == 20 || b == 5 // falskt

b > a && a < b // sant
b < a && a > b // falskt

!(a == 0 || a == 10) // sant
!(a == 5 || b == 10) // falskt
\end{lstlisting}

\begin{itemize}
    \item \&\& Ger värdet sant, ifall samtliga uttryck är sanna.
    \item $||$ Ger värdet sant, ifall ett av uttrycken är sanna.
    \item $!$ Ger värdet sant, ifall inget utav uttrycken stämmer.
\end{itemize}

\end{document}