\documentclass[../main.tex]{subfiles}

\begin{document}

\subsection{Vad menas med att "deklarera variabler"? Ge några exempel.}
Att deklarera variabler till ett program kan tolkas som att man tar fram tomma behållare man tänkt förvara kommande data i vid ett senare skede i programmet. Som i vardagslivet, dukar man fram tallrikar och glas innan maten är färdiglagad.\\
\\
Exempel;
\begin{lstlisting}[language=java]
/** I formen 'dataTyp variabelNamn;'
  * datatypen anger vilken sorts data variabeln
  * kan tilldelas.
  * datatyper;
  * int tilldelas heltalsvariabler.
  * double tilldelas flyttal (1.3, 25.5, 0.032 etc)
  * String tilldelas texter*/
  
// Deklarerade variabler  
int age, personalCodeNumber;       // Tilldelas heltal.
double height, width, thickness;  // Tilldelas flyttal.
String fullName;                   // Tilldelas texter.
\end{lstlisting}
\end{document}