\documentclass[../main.tex]{subfiles}

\begin{document}

\section{Referenser}

\subsection{C-uppgifter}

\subsubsection{Kompileringsfel}
\label{sec:compile}
\begin{lstlisting}[language=c]
#include <stdio.h>
#include <math.h>

int main ()
{
    int value;
    
    printf("Assign value: ");
    scanf("%d", value); 
    
    double root = sqrt(value);
    
    printf("Squareroot of %d = %f\n", value root);
}
\end{lstlisting}

\begin{itemize}

    \item Programmet ovan tilldelas ett heltalsvärde från användaren genom standardbiblioteket \texttt{stdio}'s funktion \texttt{scanf()};
    
    \begin{lstlisting}[language=c]
    int value; // Deklarerar heltals-variabeln, 'value'.
    
    printf("Assign value: "); // utrskrift.
    scanf("%d", value);  // Tilldelar ett heltal till variabeln.
    \end{lstlisting}
    
    \item Sedan tilldelas en flyttals-variabel, kvadratroten ur heltalsvariabeln användaren tilldelade. Genom standard funktionen \texttt{sqrt()} taget ifrån \texttt{math.h}.
    
    \begin{lstlisting}[language=c]
    double root = sqrt(value);
    \end{lstlisting}
    
    \item Och slutligen skriver ut svaret till användaren genom \texttt{printf()} funktionen från \texttt{stdio.h}
    
    \begin{lstlisting}[language=c]
     printf("Squareroot of %d = %f\n", value root);
    \end{lstlisting}
\end{itemize}

\begin{tcolorbox}[colback=green!5!white,colframe=green!75!black]
  Gå till nästa uppgift, 'logiska fel' \hyperref[sec:c2]{\textbf{här}}
\end{tcolorbox}

\end{document}