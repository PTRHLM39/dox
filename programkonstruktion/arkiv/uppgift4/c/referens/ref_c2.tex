\documentclass[../main.tex]{subfiles}

\begin{document}

\subsubsection{Logiska fel}
\label{sec:logic}
\begin{lstlisting}[language=c]
#include <stdio.h>
int main() 
{
  int n, counts = 0;

  printf("Enter value n: ");
  scanf("%d", &n);

  for (int i=1; i<=n; i++) {
  // Check if prime.
  _Bool is_prime = 1;
  for (int k = 2; k<i; k++)
    if (i % k == 0)
      is_prime = 0;
    if (is_prime) {
      counts++;
      printf("  %d", i);
      if (counts % 10 == 0)
        printf("\n");
    }
  }
}
\end{lstlisting}

\begin{itemize}

    \item I programmet ovan, tilldelas heltalsvariabeln \texttt{n} av användaren.
    \texttt{n} sätts sedan in som vilkorsdel i en \texttt{for}-sats.
    
    \begin{lstlisting}[language=c]
    for (int i=1; i<=n; i++)
    \end{lstlisting}
    
    \item \texttt{for}-loopen ovan, räknar sedan från 1 till värdet i \texttt{n}.
    För varje varv utförs även en kapslad \texttt{for}-loop deklarerad inuti \texttt{for}-loopen, ovan.
    
    \item Den kapslade \texttt{for}-loopens uppgift är att undersöka ifall värdet är ett primtal eller inte, genom iteration, moduleras värden \texttt{i} från huvud-loopen med \texttt{k} från sub-loopen.
    
    \item Om värdet \texttt{i} är delbart med \texttt{k}, sätts $is\_prime$ till falskt, och sant om resten är större än noll.
    
    \item Slutligen skrivs primtalet ut och om varven(\texttt{counts}) av \texttt{for}-loopen är delbart med 10 utförs en ny rad. Främst för att utskriften skall bli nätt.
    
\end{itemize}

\begin{tcolorbox}[colback=green!5!white,colframe=green!75!black]
  Gå till nästa uppgift, 'exekveringsfel' \hyperref[sec:c3]{\textbf{här}}
\end{tcolorbox}

\end{document}