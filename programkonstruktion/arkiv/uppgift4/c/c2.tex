\documentclass[../main.tex]{subfiles}

\begin{document}

\subsection{Logiska fel}
\label{sec:c2}
Logiska fel uppstår när man tänkt fel vid konstruktionen av programmet. Dessa fel är svåra att hitta, då programmet lyckas kompileras och exekveras utan några felmeddelanden.
Men vid test av programmet visar sig resultatet inkorrekt.\\
\\
Exemplet nedan är ett program som skall skriva ut alla primtal mellan 1 till n.
\begin{lstlisting}[language=c]
#include <stdio.h>
int main() 
{
  int n, counts = 0;

  printf("Enter value n: ");
  scanf("%d", &n);

  for (int i=1; i<=n; i++) {
  // Check if prime.
  _Bool is_prime = 0;
  for (int k = 2; k<i; k++)
    if (i % k == 0)
      is_prime = 1;
    if (is_prime) {
      counts++;
      printf("  %d", i);
      if (counts % 10 == 0)
        printf("\n");
    }
  }
}
\end{lstlisting}
Programmet gör inte som det var tänkt. Då programmet skriver ut alla tal förutom primtalen?
Med utskriften:
\begin{lstlisting}
Enter value n: 100
  4   6   8   9   10  12  14  15  16  18
  20  21  22  24  25  26  27  28  30  32
  33  34  35  36  38  39  40  42  44  45
  46  48  49  50  51  52  54  55  56  57
  58  60  62  63  64  65  66  68  69  70
  72  74  75  76  77  78  80  81  82  84
  85  86  87  88  90  91  92  93  94  95
  96  98  99  100
\end{lstlisting}
\newpage

Zoomar vi in på undersökningen av primtalen, 
\begin{lstlisting}[language=c]
for (int i=1; i<=n; i++) {
  // Check if prime.
  _Bool is_prime = 0; // ger ut 'falskt' vid primtal
  for (int k = 2; k<i; k++)
    if (i % k == 0)
      is_prime = 1; // 'sant' vid delbara tal.
    if (is_prime) { // <----------
      counts++;
      printf("  %d", i);
      if (counts % 10 == 0)
        printf("\n");
    }
  }
\end{lstlisting}
ser vi att det blivit fel vid de logiska sannings-värdena.\\
\\
Genom att ändra de logiska uttrycken till att istället ge ut 'sant' vid primtalen
\begin{lstlisting}[language=c]
_Bool is_prime = 1; // 'sant' vid primtal
.....
.....
if (i % k == 0) 
  is_prime = 0; // 'falskt' vid delbara tal
if (is_prime){    // 'sant' vid primtal
....
}
\end{lstlisting}
Ger sedan korrekta utskriften:
\begin{lstlisting}
Enter value n: 100
  1  2  3  5  7   11  
  13  17  19  23  29
  31  37  41  43  47  
  53  59  61  67  71
  73  79  83  89  97
\end{lstlisting}
\begin{tcolorbox}[colback=green!5!white,colframe=green!75!black]
Referens till programtext finner du \hyperref[sec:logic]{\textbf{här}}
\end{tcolorbox}
\end{document}