\documentclass[../main.tex]{subfiles}

\begin{document}

\subsection{\textit{Beskriv skillnaden mellan lågnivåspråk (assembleringsspråk) och högnivåspråk? Ge exempel.}}

\texttt{Lågnivåspråk beskriver vad datorns processor skall göra, högnivåspråk beskriver vad programmet skall göra och låter sedan en kompilator översätta källkoden till instruktioner som datorns processor kan exekvera.}\\
\subsubsection{\textsc{Lågnivå}}
\texttt{Assembleringsspråk är tämligen lik maskinkoden, skillnaderna ligger vid att instruktionerna har namngivits i assembler, för att göra det mer läsbart för människor.\\ Se exempel för lågnivåspråk nedan.}


\begin{lstlisting}
; "Hello, World" i x86 assembler 

          global    _start

          section   .text
_start:   mov       rax, 1                  
          mov       rdi, 1                  
          mov       rsi, message            
          mov       rdx, 13                 
          syscall                           
          mov       rax, 60                 
          xor       rdi, rdi                
          syscall                           

          section   .data
message:  db        "Hello, World", 10   
\end{lstlisting}

\subsubsection{Högnivå}
\texttt{Det kallas högnivåspråk i och med att abstraktionsnivån är högre, vilket gör saken betydligt lättare för människor att förstå. \\Nedan är ett exempel på högnivåspråk.}

\begin{lstlisting}[language=c]
/*"Hello world" i C*/
#include <stdio.h>

int main() {
  printf("Hello, World\n");
  return 0;
}
\end{lstlisting}

\end{document}