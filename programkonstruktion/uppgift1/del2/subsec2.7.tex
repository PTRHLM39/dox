\documentclass[../main.tex]{subfiles}

\begin{document}

\subsection{\textit{Vad har kompilatorn för uppgift? Beskriv kompileringsprocessen. Skriv i detalj hur man skapar och kompilerar ett java-program.}}

Kompilatorer är program, vars uppgift är att konvertera programtext skrivet i ett visst programmeringsspråk till binärkod. Dvs, instruktioner i form av
ettor och nollor som sedan kan laddas in i primärminnet och exekveras i processorer. Detta kallas kompilera.

\subsubsection{Kompileringsprocessen}

%% Flow-chart figure
\begin{wrapfigure}{l}{0.6\textwidth}

\begin{tikzpicture}[node distance=2cm]

\node (file) [file] {Källkod (.Java-fil suffix)};
\node (command) [command, below of=file] {javac kommando};
\node (file1) [file, below of=command] {Bytekod (.Class-fil suffix)};
\node (jvm) [platform, below of=file1] {V};
\node (jvm1) [platform, left of=jvm] {J};
\node (jvm2) [platform, right of=jvm] {M};

\node (os) [os, below of=jvm1] {Windows};
\node (os1) [os, below of=jvm] {Linux};
\node (os2) [os, below of=jvm2] {OSX};

\draw [arrow] (file) -- (command);
\draw [arrow] (command) -- (file1);
\draw [arrow] (file1) -- (jvm);
\draw [arrow] (file1) -- (jvm1);
\draw [arrow] (file1) -- (jvm2);
\draw [arrow] (jvm) -- (os1);
\draw [arrow] (jvm1) -- (os);
\draw [arrow] (jvm2) -- (os2);

\end{tikzpicture}

\end{wrapfigure}
%% Flow-chart end

Att skriva och kompilera ett eget java-program utförs genom att skriva programtext som följer javas uppsättning av regler. Spara filen i filnamnändelsen \textbf{.Java}. Kör sedan kommandot; \textbf{javac Filnamn.Java} i en kommandotolk.  Kollar du nu i mappen där \textbf{.Java}-filen kompilerades finns även nu en \textbf{.Class}-fil tillgänglig. Har inga felmeddelanden dykt upp i terminalen och \textbf{.Class}-filen ligger i din mapp var kompileringen lyckad. Exekvera sedan ditt program genom att köra javas virtuella maskin med kommandot \textbf{java Filnamn} i kommandotolken.

\end{document}