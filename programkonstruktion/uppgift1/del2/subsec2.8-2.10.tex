\documentclass[../main.tex]{subfiles}

\begin{document}

\subsection{\textit{Varje språk har sin egen syntax! Vad menas med "Syntax"?}}

Programmerings-syntax innehåller strängar av text liknande de mänskliga språken. Syntaxen refererar till en uppsättning regler som definerar hur ett program skrivs och tolkas. Korrekt formulerad syntax ger korrekta fraser och satser inom ett specificerat programmeringsspråk.

\subsection{\textit{Vad har Java Virtuell Maskin (JVM) för uppgift? Förklara!}}

JVM är en virtuell maskin vars uppgift är att göra java-program portabla. Med andra ord beter den sig som en komplett enskild dator i förhållande till java-program. 
Den "översätter" Java-programmen så att de sedan kan exekveras på den dator och operativsystem JVM körs på. Vilket gör att java-programmen enkelt kan köras på vilken dator och vilket operativsystem som helst utan att behöva ändra källkoden.

\subsection{\textit{Vad har länkaren för uppgift?}}

\begin{tikzpicture}[node distance=2.5cm]

\node (os) [os] {Källkod};
\node (os1) [os, left of=os] {Källkod};
\node (os2) [os, right of=os] {Källkod};

\node (javac) [command, below of=os] {Kompilator};

\node (class) [os, below of=javac] {.class};
\node (class1) [os, left of=class] {.class};
\node (class2) [os, right of=class] {.class};

\node (linker) [command, below of=class] {Länkare};

\node (exe) [os, right of=linker] {exe};

\draw [arrow] (os) -- (javac);
\draw [arrow] (os1) -- (javac);
\draw [arrow] (os2) -- (javac);

\draw [arrow] (javac) -- (class);
\draw [arrow] (javac) -- (class1);
\draw [arrow] (javac) -- (class2);

\draw [arrow] (class) -- (linker);
\draw [arrow] (class1) -- (linker);
\draw [arrow] (class2) -- (linker);

\draw [arrow] (linker) -- (exe);
\end{tikzpicture}
\\
Länkarens uppgift är att kombinera (länka) samman flera class/objekt-filer för att därigenom skapa exekverbara program.
\end{document}