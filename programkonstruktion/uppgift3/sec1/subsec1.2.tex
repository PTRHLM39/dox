\documentclass[../main.tex]{subfiles}

\begin{document}

\subsection{Beskriv den booleska datatypen och skriv ett exempel (källkod) där den används.}

Variabler kan även deklareras till datatypen \textbf{boolean}. Det här en typ som används för att beskriva sanningsvärden.
\\
En \textbf{boolean} kan enbart innehålla två olika värden, dessa värden är \textbf{true} eller \textbf{false}.
\\
I programmering är det vanligt att man behöver en typ som kan innehålla en av två värden, som:
\begin{itemize}
    \item Ja / Nej
    \item På / Av
    \item Sant / Falskt
\end{itemize}

\begin{lstlisting}[language=java]
// Exempel
boolean isJavaFun = true; // Boolean initiering

if (isJavaFun)
    System.out.println("Java is fun!");  // Utmatning
else
    System.out.println("Java is not fun..");
\end{lstlisting}
\end{document}