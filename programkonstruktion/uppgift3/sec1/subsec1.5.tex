\documentclass[../main.tex]{subfiles}

\begin{document}

\subsection{Förklara vad ökningsoperatorn (++) och minskningsoperatorn (- -) gör med hjälp av några exempel (källkod).
Dessa operatorer finns i två varianter, prefix och postfix.
Förklara också skillnaden mellan dessa två varianter med hjälp av ett exempel (källkod).}

När man utför upprepningar i program förekommer det ofta man vill öka värdet av en tilldelningsvariabel som sätts som räknare. Där med kommer öknings och minskningsoperatorer väl till pass.

\subsubsection{Ökning och minskningsoperatorn.}
\begin{itemize}
    \item ++variabelNamn ökar önskad heltalsvariabel med 1.
    \item - -variabelNamn minskar önskad heltalsvariabel med 1.
\end{itemize}
Nedan är ett program som räknar från 0 upp till 5 och sedan ner igen från 5 till 0 med hjälp av dessa operatorer.
Operatorerna är definerade i \textbf{for}-loopernas förändringsdel.
\begin{lstlisting}[language=java]
public class Increment
{
   public static void main(String[] args)
   {
      String a = "incriment: ", b = "decriment: ";
      
      for (int i = 0; i < 5; ++i)
      {
         System.out.println(a + i);
         
         if (i == 4)
         {
            for (int d = 5; d >= 0; --d)
            {
               if (d == 5)
                  System.out.println("Peek! Going down..");
               else
                  System.out.println(b + d);
            }
         }
      }
   }
}
\end{lstlisting}

\newpage

\begin{lstlisting}[language=java]
/*Utmatning:*/
incriment: 0
incriment: 1
incriment: 2
incriment: 3
incriment: 4
Peek! Going down..
decriment: 4
decriment: 3
decriment: 2
decriment: 1
decriment: 0
\end{lstlisting}

\subsubsection{Prefix och postfix.}
Prefix(++variabelNamn) och postfix(variabelNamn++) är väldigt lika, men inte riktigt desamma. Hade postfix används till exemplet ovan hade det inte blivit någon skillnad i resultatet. Båda ökar (och med - -, minskar) variablernas värden. Prefix (++i) ökar värdet \textbf{innan} det aktuella uttrycket utvärderas, medan postfix(i++) ökar värdet \textbf{efter} att uttrycket har utvärderats!
\\
Exempel 
\begin{lstlisting}[language=java]
public class Postfix 
{
   public static void main(String[] arg)
   {
      int x, y, z, a, b, c;
      
      x = 8; y = 6;
      // Prefix 
      z = ++x * --y;
      System.out.println(z); // 45 = 9 * 5
      
      a = 8; b = 6;
      // Postfix
      c = a++ * b--;
      System.out.println(c); // 48 = 8 * 6
   }
}
\end{lstlisting}
\end{document}