\documentclass[../main.tex]{subfiles}

\begin{document}

\subsubsection{Indata}
Tanken är att låta programmet tilldelas personnummer, för att ge informationen vi behöver.\\
\\
\begin{algorithm}[H]
\SetAlgoLined

\KwResult{Tilldela programmet en sträng-variabel innehållande personnummer med formen: ÅÅÅÅMMDD-NNNN }

 \textbf{Sträng} personNummer\;
 
 \textbf{Boolean} korrekt = personNummer == ÅÅÅÅMMDD-NNNN \;
 
 \While{korrekt}{
  Print "Något gick fel, försök igen."\;
 }
 Programmet fortsätter
 \caption{\textbf{Indata.} Be användaren om personnummer.}
 
\end{algorithm}

\begin{algorithm}[H]
\SetAlgoLined

\KwResult{Returnerar personens ålder som heltal}

 \textbf{Heltal} födelseÅr, ålder\;
 födelseÅr = ÅÅÅÅ\;
 ålder = nuvarandeÅr - födelseÅr\;
 \caption{\textbf{Indata.} Tar födelseåret ur personnumret som argument och beräknar personens ålder.}
\end{algorithm}

\begin{algorithm}[H]
\SetAlgoLined

\KwResult{Returnerar personens kön som boolean}

 \textbf{Heltal} könSiffra\;
 \textbf{Boolean} kön\;
 könSiffra = NNNN\;
 kön = Sant om värdet i 'könSiffra' är jämn, annars falskt om udda\;
 \caption{\textbf{Indata.} Tar tredje siffran ur löpnumret(NNNN) från personnumret som argument och beräknar personens kön.}
\end{algorithm}

\newpage

\end{document}